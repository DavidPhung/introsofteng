%
% Copyright 2017 Markus Borg, Lund University
%
% This work is licensed under a Creative Commons Attribution-ShareAlike 4.0 International License.
% See http://creativecommons.org/licenses/by-sa/4.0/
%
% The dodument is based on a LaTeX template developed by Jean-Philippe Eisenbarth
% https://github.com/jpeisenbarth/SRS-Tex
%
\documentclass{scrreprt}
\usepackage{listings}
\usepackage{underscore}
\usepackage[bookmarks=true]{hyperref}
\usepackage[utf8]{inputenc}
\usepackage[english]{babel}
\hypersetup{
    bookmarks=false,    % show bookmarks bar?
    pdftitle={Lab 1},    % title
    pdfauthor={Markus Borg},                     % author
    pdfsubject={TeX and LaTeX},                        % subject of the document
    pdfkeywords={TeX, LaTeX, graphics, images}, % list of keywords
    colorlinks=true,       % false: boxed links; true: colored links
    linkcolor=blue,       % color of internal links
    citecolor=black,       % color of links to bibliography
    filecolor=black,        % color of file links
    urlcolor=purple,        % color of external links
    linktoc=page            % only page is linked
}%
\def\myversion{0.1 }
\date{}
%\title
\usepackage{hyperref}
\begin{document}

\begin{flushright}
    \rule{16cm}{5pt}\vskip1cm
    \begin{bfseries}
    	\LARGE{ETSA02-ADM-LAB1}\\
    	\vspace{1.5cm}
        \Huge{Lab 1}\\
        \vspace{0.5cm}
        Getting started\\
        \vspace{0.5cm}
        with Robocode\\
        \vspace{1.5cm}
        \LARGE{Version \myversion approved}\\
        \vspace{1.5cm}
        Prepared by Markus Borg\\
        %\vspace{1.5cm}
        Dept. of Computer Science, Lund University\\
        \vspace{1.5cm}
        \today\\
    \end{bfseries}
\end{flushright}

%\tableofcontents

\chapter*{Revision History}

\begin{center}
    \begin{tabular}{|c|c|c|c|}
        \hline
	    Name & Date & Reason For Changes & Version\\
        \hline
	    Markus Borg & 2017-12-08 & Initial draft. & 0.1\\
        \hline
    \end{tabular}
\end{center}

\chapter{Introduction}
Learn to install and configure a non-trivial software framework by following instructions.
Learn the basics of Robocode.
Use Eclipse to build a basic robot.
Configure Robocode to import new robot.
Understand the basics of teams in Robocode.

\chapter{Before the lab}
Read up on the basics of Robocode.
Have a look at the API.

\chapter{At the lab}

\section{Install Robocode}
Download latest Robocode: https://sourceforge.net/projects/robocode/files/
Install on local drive
Start a few battles. Experiment.
Develop a quick robot using the internal editor: http://robowiki.net/wiki/Robocode/My_First_Robot
Compile it and try it in battles.

\section{Develop a Robot in Eclipse} \label{sec:develop}
Create a project in Eclipse http://robowiki.net/wiki/Robocode/Eclipse/Create_a_Project
Download the Basic Leader Bot to the project. Make sure it builds.

\section{Try the robot in Robocode} \label{sec:try}
Add the Robot project in Robocode: http://robowiki.net/wiki/Robocode/Add_a_Robot_Project
Note: Add the path to the Eclipse project (with /src and /bin as sub-folders), not the workspace.

\section{Expand the team} \label{sec:expand}
Repeat the instructions in Sections~\ref{sec:develop}-\ref{sec:expand} for the Basic Bot and the Basic Drone. Try a few team battles against various opposition.

\chapter{After the lab}
Reflect on what is important for successful team battles. Consider your two perspectives. First, think about what type of team you want to build to be run a successful LU rumble.

Second, discuss what type of Robot you want to release on the market. What types will be in high demand? What is the competition going to offer? What could be your niche? Think about commodity, differentiation, and innovation.

Successful development of software products for an open market requires making critical decisions under time pressure -- decisions of both technical nature as well as business nature. Your team needs to decide soon!

\end{document}
